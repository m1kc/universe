\documentclass[12pt,a4paper]{book}
\usepackage[utf8]{inputenc}
\usepackage[russian]{babel}
\usepackage{amsmath}
\usepackage{amsfonts}
\usepackage{amssymb}

\title{\textbf{uStory}}
\author{Автор: АндРЕЙ \and Координатор: m1kc}

\begin{document}

\maketitle

\chapter*{Пролог}

Эрия... Именно так называется мир, о котором пойдёт речь. Это огромный материк, на котором живут перворожденные, урук-хай и (куда же без них) люди. О перворожденных ни мне, ни расе людей почти ничего не известно. Народу урук-хай тем более, орки издавна мало интересовались окружающими, питая какую-то непонятную неприязнь к чужакам, как они называли людей и перворожденных. Надо сказать, что чужаками были скорее они, так как по хроникам эльфов орки лет на 500 позже людей высадились на берегу Великого океана. И сразу, даже не успев обжиться на новом месте, объявили войну \mbox{эльфам ---} якобы глас богов повелевал уничтожить презренных остроухих. Как не странно, люди поддержали их, обратившись против своих наставников. Что было тому виной --- алчность ли правителей, охочих до сокровищ перворожденных, желание ли показать мощь новоявленной имперской армии? Доподлинно неизвестно, да и теперь уже неважно. Эльфы не смогли долго сдерживать объединённые силы людей и урук-хай и отступили в глубину Вечных лесов. Отряды людей (а урук-хай сразу сказали, что носу не сунут в эти проклятые Великие заросли), решившиеся преследовать перворожденных, бесследно исчезли, назад не вернулся ни один воин. Тогда было принято решение прекратить атаку и возвести вокруг южной части леса стену с дозорными башнями. Мол, пусть сидят и не высовываются. И на долгие века люди забыли о перворожденных... Да и с орками стали общаться гораздо меньше, торгуя лишь по большой необходимости. Тем более, что по окончанию войны с эльфами империи пришлось заняться собственными внутренними проблемами --- высокородные лорды раздирали государство на куски, народ роптал, осознав бесполезность войны, унесшей тысячи людей, да и церковь, основательно прижатая прошлым правителем, вновь набирала силу, что было чревато большим бунтом.

Урук-хай тоже постарались обособиться от людей --- слишком разной была культура и религия. Да и общественное устройство орков, советы кланов, старейшины, руководствующиеся не столько политическими соображениями, сколько <<волею богов>>, непонятно (с точки зрения имперцев, разумеется) в чем проявляющейся, были слишком чужды людям.

И понеслись дни, месяцы, века... Империя стала монархией, потом развалилась на клочки уездов, потом опять монархия, тирания, демократия, монархия...

А вот здесь, пожалуй, стоит остановиться, ибо после случилось событие, изменившее дальнейшую судьбу огромной страны. Надо сказать, что люди до оного времени не имели ни малейшего понятия о магии, и магов среди них не было. Да, в давние времена видели, как перворожденные исцеляли людей, но сами ничего подобного не умели. И вдруг, через 800 эльфийских циклов после войны (7000 лет), объявился человек, объявивший, что владеет магическими способностями. Поначалу, разумеется, его подняли на смех и даже чуть не изваляли в дегте. Тогда он превратил стоящий на площади дом в цветущее дерево, которого никто и не видел. Впрочем, он недолго показывал фокусы на потеху толпе. Уже на следующий день великий князь Та-Реми, выйдя на трясущихся ногах на балкон в сопровождении мага, объявил во всеуслышание, что отказывается от всех прав на престол в пользу этого самого мага, Ремана I. А тот сообщил ошеломлённым жителям Трена, столицы империи, что вводит новый тип государственного устройства --- ненаследственную монархию, суть которой заключается в том, что преемника себе правитель находит, ориентируясь лишь на его моральные качества. Правителем может стать кто угодно --- от городского старосты до девушки из соседнего кабака. И никто, даже родные, не будут знать императора в лицо. Личность правителя становится известна лишь после его смерти. Примерно такой была суть пространной речи новоявленного императора.

Собравшиеся на площади люди поначалу ничего не поняли, но единодушно решили, что по такому поводу можно и напиться.

Впрочем, простой народ с облегчением вздохнул уже через пару месяцев --- стали открываться новые лечебницы и школы, снизились налоги. Хотя главным событием было бесспорно другое --- новый император объявил, что некоторые люди способны творить магию, и открыл специальные гильдии, которые занимались набором и обучением наделенных даром детей. Через какие-то двести лет никто уже не мог представить обыденную жизнь без магии, относясь к ней, как к чему-то совершенно естественному. Другое дело церковь. Первосвященник объявил магов пособниками дьявола и запретил всякое общение с ними. Но гнев \mbox{Благих ---} это ещё когда будет, а польза от магии --- вот она, под носом. Церковники быстро поняли это и на время затихли. Жизнь в империи потекла своим, хотя и новым, чередом.

\chapter{Грёбаная деревня}

\section{Не очень радостное утро}

Герой просыпается.

--- Оох... Голова раскалывается... Хвост проклятого мне в глотку...зарекся ведь не пить больше... Так, а по какому поводу это я так? Поляна, староста, бочонки...танцы, толпа народу...потом Энди...или как там ее звали? А-а, празднество, ночь Всех Благих. Ну по такому поводу выпить можно было... Хотя...не надо было...

--- Так, а где я собственно есть? Надо бы найти этого дармоеда, целителя нашего деревенского. Увалень никчемный, но похмелье сбивать на раз умеет. Этой практики у него тут хватает.

\section{Не очень приветливый целитель}

У целителя.

--- Доброе утро, Тирек. Мне тут нужна кое-какая помощь... 

--- Угу, кто бы сомневался. У меня сегодня таких как ты уже с полсотни побывало. Ладно, это все пустое. Пятнадцать дени --- и ты как огурчик.

1). --- Это, подожди, какие деньги? Мы ведь друзья? 

--- Да, сегодня все вы тут друзья. Мои услуги стоят пятнадцать дени.

1.1). По ветке 2.

1.2). По ветке 3. 

2). Идет. Скоро вернусь с деньгами.

- Если на мели, советую зайти к старейшине, у него должна найтись работенка даже для...гм...в общем, ты тоже сможешь заработать. 

3). А нельзя ли договориться каким-то другим образом?

- Хм... Мне нужно доставить эту посылку...э-э...знакомому. Только учти, что моего имени ты ему говорить не должен. Вообще ничего обо мне. Это очень важно. Согласен?

3.1). - Да

- Хорошо. Разумеется, пойдешь ты не в таком состоянии. Выпей вот это.

3.2). - Нет

- Тогда напоминаю, пятнадцать дени. И не дай Благие что бы кто-то узнал о моем предложении.

3.2.1). - Никому не говорить? Без проблем. Сними похмелье. И, пожалуй, пятнадцать дени впридачу, так забавнее.

- Сволочь пьяная! Вот деньги и снадобье. Убирайся!   

3.2.2). Все просто - ты мне помогаешь, я ухожу и забываю наш разговор.

- Наглый скот! Вот снадобье, убирайся.

3.2.3). По ветке 2.

\section{Не очень забавная встреча}

Если игрок берет задание.

- Добрый день.

- Очень. Цель визита?

1). - Да вот встретил тут неподалеку какого-то таинственного мужчину. Он просил передать вам это. Сказал, что вы заплатите.

- Что за черт?(распаковывает)

Ах сволочь, проклятый тебя побери! Стоять!

- Э-э...но мне пора.

- КТО ТЕБЕ ЭТО ДАЛ? Точнее, где он? Прежде чем соврать, подумай.

1.1). - Я все понял. Эту посылку дал мне Тирек, целитель из столицы.

- Из столицы говоришь... Что ж, думаю, не врешь. Можешь идти.

1.1.1). - А вознаграждение за услугу?

- Слушай, в первый раз вижу столь наглого идиота. Но чем-то ты мне нравишься. Вот 20 дени. Проваливай.

1.1.2). - ...(уйти)

1.2). - Говорю же, незнакомец какой-то на безлюдной дороге.

- Я маг. И знаю, что сейчас этот человек в каком-то людном месте. Больше я сам узнать, к сожалению не в состоянии, но это неважно. Ты солгал.

(бой)

2). - Да вот встретил тут неподалеку какого-то таинственного мужчину. Он просил передать вам это.

[по первому диалогу без ветки 1.1.1]

3). Некто Тирек, целитель из столицы просил...

- ПЛЕВАТЬ. ГДЕ ОН?

3.1). - В небольшом селении [там-то там-то]

3.2). - Любая информация имеет свою цену...

- Три-четыре взмаха меча, и ты больше никогда не станешь искать выгоду там, где ее нет. Ну, что скажешь?

3.2.1). по ветке 3.1

3.2.2). - Проверим.

(бой)

3.2.3). - Мечом, говоришь, помахаешь? А мне кажется, что нет. Тебе нужна эта информация, значит, ты за нее заплатишь. Мой труп тебе ничего не сможет сказать.

- Шельма... Ты прав. И у меня нет времени пытать тебя. Вот 150 дени, больше у меня нет.

3.2.3.1). - Что ж, это другой разговор.+ по ветке 3.1

3.2.3.2). - Это слишком несерьезная сумма.

- Проходимец... Мы не на рынке. Убью тебя, а потом уже поищу Тирека Варха.

(бой)  

4). - Да гулял тут неподалеку, вижу, хижина одинокая. Решил зайти.

- Выметайся, пока жив.

5). - Набить тебе морду.

- Да? Большая ошибка, сосунок.

\section{Передышка}

Если игрок не берет задание целителя, то новая цель - заработать денег.

Если игрок сдает целителя, то больше в селе его не будет.

Если игрок не сдает целителя, то кроме торговых вариантов дальнейшего диалога с ним должен быть вариант:

- Я передал посылку.

- И не сказал, от кого она?

- Нет разумеется.

- И жив?? А, о чем это я... Я занят, не отвлекай.

После избавления от похмелья/выполнения задания целителя должен быть монолог:

- Да, чуть не забыл, батя же просил к нему утром зайти... Надо бы поторопиться.

\section{Не очень аккуратный почерк}

В доме отца.

- И где он, интересно? Ладно, пока его жду можно тут осмотреться.

[письмо]

Здравствуй, сын мой. Произошли некоторые неприятные вещи и я должен был отлучиться. Если ты читаешь это письмо, значит, меня задержали некоторые обстоятельства. Да, должен сказать тебе, что я не только простой кузнец, каким ты меня всегда считал. Остальное не могу доверить бумаге. Найди в Трене Керри Руфельдсона, он расскажет тебе больше. Да, и будь с ним поосторожнее. Этот неприметный человек весьма непрост. Хоть он и знает, что ты мой сын, но все же... Удачи.

Все вышесказанное - срочно!

P.S. Найти Керри не так просто, у него весьма...специфичные связи. Но, думаю, кто-то из богатых торговцев должен о нем знать. Еще можешь порасспрашивать завсегдатаев трактиров низкого пошиба, только лишнего там не болтай.

\chapter{Долгая дорога в Трен}

\section{Лёгкие неприятности}

[где-то по пути к городу должно быть что-то вроде лагеря с 5-7 людьми. Герой может с ними встретиться - состоится следующий диалог]

1). - Здравствуйте.

- Что надо?

1.1). Да ничего, просто мимо прохожу.

- Ну так проходи, и побыстрее.

1.2). А кто вы, собственно, такие?

- Разве непонятно? Высокородные лорды собрались на полянке. Ха-ха-ха. Мы наживаемся на таких как ты, придурок.

1.2.1). - Я, пожалуй, пойду.

- Прошу вас, господин, не торопитесь. Вы кое-что забыли, хех.

(бой)

1.2.2). Полегче, милейший.+ по ветке 1.3

1.2.3). - Я бы сказал, наживались.

(бой)

1.3). - Мне нужны толковые люди, умеющие держать в руках оружие. Случайно заметил вас.

- Интересно... Что за дело, какова плата?

- Плачу щедро, не беспокойтесь. Дела самые разные, но...гм...вы в таких разбираетесь.

- Хорошо. Если мы понадобимся, найдешь нас здесь.

2). - Как-то подозрительно вы выглядите...

- Глубокая мысль, ничего не скажешь. Что тебе надо?

2.1). - Да думаю вот, не заинтересуются ли вами стражники ближайшей деревеньки... Есть выгодное предложение - вы мне даете звонкий мешочек, я о вас забываю.

- Зачем такие сложности, милсдарь? Есть же другой выход... 

(бой)

2.2, 2.3 и т.д. - все варианты первого диалога.

3). - Что-то вы мне не нравитесь...

- Какое совпадение. Ты нам тоже. А с чего такой смелый, парень? Из горных мастеров? Или, может паладин Святой церкви, а? Ха-ха... Посмотрим.

(бой)

\section{Забавные стражники}

Около городских ворот стоит стража.

- Эй, парень, куда собрался?

1). - В город.

- Нельзя в город. Тут неподалеку в деревушке эпидемия какая-то объявилась, приказали закрыть все ворота. Да, и близко к нам не подходи, а то мало ли...

1.1). - А не помогут ли маленькие золотые кругляши забыть об этом?

- 1000 дени.

1.1.1). - Идет.

1.1.2). - Увидимся позже.

1.2). - Видите ли, мне очень нужно попасть в город. А посему лучше бы вам не чинить мне препятствий. В противном случае трое из вас умрут наверняка, еще двое - если мне чуть-чуть повезет. Итак, кто хочет войти в гарантированную тройку призеров?

[смерть]

1.3). [вернуть к лагерю разбойников и договориться с ними о нападении на стражу] (если бравый герой этих разбойников еще не перебил)

1.4). - Послушайте, уважаемые, но должен же быть какой-то способ попасть внутрь?

- Хех, есть-то есть, но сомневаюсь, что ты сможешь им воспользоваться. 800 дени за проход и немедленное обследование в гильдии целителей.

1.4.1). - Мне это подходит.

1.4.2). - Увидимся позже.

1.5).[лидерство] - Эпидемия... Как не вовремя... Дело в том, что я срочно должен попасть в город, мне назначена встреча, которую я не могу проигнорировать.

- Ахаха... Важная встреча... Так бы и сказал, что девчонка смазливая приглашает. Хах. Ладно, только из мужской солидарности. Проходи. Но в гильдию целителей потом все же наведайся, а то мало ли... Разумеется, это я про эпидемию, ха-ха-ха.

2). - Ты совсем идиот? В город, разумеется.

- Эпидемия, проход закрыт.

2.1, 2.2 и т.д. из первого диалога со следующими изменениями:

Ответ на 1.1 - Нет. Не люблю заносчивых идиотов.

3).[при достаточном красноречии, которое харизма в твоем списке :-) ] - Видите ли, уважаемые, я писатель. Сейчас создаю новое произведение о будничной работе нашей доблестной стражи. Вы, например, ничего не хотите мне рассказать о себе?

- Я...э-э...да, точно. Только это...мне надо собраться...значит, с мыслями.

- Сожалею, но я не могу ждать этого, стоя на дороге. Давайте встретимся позже в каком-нибудь трактире, где я смогу увековечить ваши деяния?

- Так оно это, приказ отдан никого не впускать, будь он не ладен...

А, была не была. Проходи, только не болтай там об этом. 

- Лерн, ты балда? Знаешь, что с тобой командир поста сделает, коли узнает?

- А от кого он узнает-то, собака, как не от вас? Ты, Огар, трепи меньше, все и обойдется.

- В таком случае, всего наилучшего, мне пора.

- Эй, а где ты меня найдешь?

- Тьфу, пропасть. Да, разумеется. Надеюсь на встречу с вами в трактире NN.

\section{Уютные трактиры}

Город.

A). Если герой ищет того самого Керри по трактирам. Да и вообще, возможные ответы трактирщиков в любой момент игры.

- Прекрасный день, господин. Что вам угодно?

/Другие варианты

- Польщен вашим присутствием в моем скромном заведении. Что пожелаете?

/ - Чем обязан?

1.1). - Я хочу снять комнату. Есть у тебя свободные?

- Комнаты есть. Но, прошу простить меня за откровенность, по карману ли они вам будут? В городе есть постоялый двор, там подешевле будет...

1.1.1). - Подешевле? Ты за кого меня принимаешь, скот? Да я твой паршивый трактир купить могу!

- Прошу прощения, сиятельный господин. Комната стоит 10 дени в день. Или 250 за месяц. Займете?

 - Да.

 - Гм... Пожалуй, не сейчас.

/ другой ответ на 1.1.

- Да, разумеется. Прикажете приготовить? Ах да, оплата... 12 дени в день.

- Мне это подходит.

- Возможно, позже.

/ Свободных нет.

/ Я не сдаю комнаты. Тут не постоялый двор.

1.2). - Послушай, милейший, я ищу одного человека. Зовут Керри Руфельдсон. Не слыхал о таком?

- К моему глубочайшему сожалению, нет. Что-нибудь еще?

/- Ничем не могу помочь. Я не знаком с этим господином.

/- Откуда узнал обо мне? Место глухое, нигде не бываю... А зачем он тебе?

1). - Э-э...навести кое-какие справки.

- Справки? У Керри? Ха-ха. Да знаешь ли ты, кто он такой?!

1.1). - Знаю. И повторяю, мне нужно встретиться с ним.

- Я переговорю с ним. Если Керри изъявит желание тебя видеть, то сам найдет.

1.2). - Кто?

- Что-то ты темнишь, парень. Забудь о нашем разговоре.

[провал. Игрок должен будет вести поиски по второй ветке] 

2). - По личному делу. Понимаешь, о чем я?

- Я расскажу ему о твоем визите. И если Керри заинтересуется, то сам тебя найдет. 

3). - Тебя это не касается. Где его найти?

- Неважно. Я расскажу о твоем визите. Если Керри заинтересуется, то сам тебя найдет.  

1.3). - Налей-ка мне что-нибудь.

- Что желаете?

[односторонняя торговля]

1.4). - Подай обед.

- Сию секунду.

[восстановление здоровья]

1.5). - Что слышно в городе?

[эти реплики, так же как и реплики npc, не участвующих в сюжете, буду делать в самом конце]

B). Если герой пытается получить информацию у торговцев.

- Добро пожаловать, юноша. Лучшие товары для вас. Ну, разумеется, в случае вашей платежеспособности. Это так, к слову.

/Приветствую! Прошу, выбирайте.

/ Добрый день, юноша. Желаете сделать покупку? Я к вашим услугам.

/ Рад видеть вас. Желаете посмотреть товары? Прошу.

/ О, новый клиент. Только у нас лучшие товары по хорошим ценам!

1). - Посмотрим, посмотрим. [торговля]

2). - Хотелось бы занять у вас денег в долг...

- Разумеется. Но учти, условия займа зависят от твоей репутации в нашей гильдии. Да, и просто на всякий случай... Не пытайся меня обмануть, в противном случае тобой...простите, вами займется стража. Причем не только в нашем городе.

/ Мое золото к вашим услугам. Но...не забудьте вернуть его вовремя...

3). - Меня интересуют не столько товары, сколько информация. Я ищу некоего Керри Руфельдсона. Знаешь о нем что-нибудь?

- Керри, Керри... А, припоминаю, есть тут пьянчужка один. Только он Керри Дарин. А Руфельдсона... Нет, не знаю.

/ - Вот так-так... Я...гм...нет. Не торгую такой информацией, себе дороже станет.

3.1). - [лидерство] Поверьте, у вас не будет никаких неприятностей. Он мой закадычный приятель. Все, что меня интересует - личная встреча.

- Мда... Получается, я виноват в любом случае. Скажу, где его найти - так его люди с меня голову снимут. Не скажу - тоже, если ты действительно его приятель...

- Есть и другой вариант, вероятно, более приемлемый. Вы мне ни слова не говорите и сами передаете Керри наш разговор. После чего он сам со мной встречается в заранее оговоренном месте.

- Хорошо, я передам. Поверьте, если он заинтересуется, то найдет вас где угодно. Я передам. Но вы забыли кое-что.

- Да?

- 500 дени за посредничество.

3.1.1). - Идет.

3.1.2). - 300 - более разумная цена за эту скромную услугу.

- Господин, здесь торг не уместен.

- Хорошо, сойдемся на пятистах дени.

3.2). - Жаль, жаль...

[еще один ответ на 3]

- Предположим, я владею некими сведениями об этом господине. Что конкретно вас интересует?

- Встреча.

- Вот как... А будет ли заинтересовано в этой встрече названное вами лицо?

- Думаю, да.

- В таком случае, я передам ему этот разговор. Если Керри захочет поговорить, то найдет вас сам. Да, чуть не забыл. Мое посредничество обойдется вам в 600 дени.

- Идет.

\section{Занятный незнакомец}

[так или иначе, но после успешного завершения задания к герою через некоторое время подбегает чел...]

- Здравствуйте. Уделите мне минуту внимания.

1). - Да, слушаю.

- Мне приказано проводить вас к человеку, которого вы искали. Следуйте за мной.

1.1). - Пройдемте.

1.2). - Э-э...у меня есть срочные дела. Возможно, позже.

- Мне придется прояснить некоторые детали... Этот человек заинтересовался вами, а это значит, что он встретится с вами в любом случае. Вопрос лишь в том, пойдете вы к нему добровольно, или же...э-э...все будет несколько иначе. Неподалеку ждут 18 отлично обученных бойцов. Надеюсь, ваши срочные дела стали менее важными?

- Гм... Да, ваши доводы весьма убедительны.

1.3). - Я передумал. Этот человек меня абсолютно не интересует.

- Зато вы его интересуете, а посему соблаговолите проследовать к нему.

1.3.1). по ветке 1.1

1.3.2). по ветке 1.2

2). - Я тороплюсь.

- Это неважно. Мне приказано проводить вас к человеку, которого вы недавно искали. Следуйте за мной.

2.1). по ветке 1.1

2.2). по ветке 1.2

3). - Да пошел ты.

- Сейчас. Причем в вашей компании.

- Чего?

- Мне приказано проводить вас к человеку, которого вы недавно искали. Следуйте за мной.

3.1). по ветке 1.1

3.2). по ветке 1.2

4). - Я тороплюсь. У тебя есть сорок секунд. Слушаю.

- Хех... Вы оценили возможную информативность моего разговора в сорок секунд вашего времени?

4.1). - Чего?

- Так, понятно все. Мне приказано проводить вас к человеку, которого вы недавно искали. Следуйте за мной.

4.1.1). по 1.1

4.1.2). по 1.2

4.2). - Именно. Итак?

- Мне приказано проводить вас к человеку, которого вы недавно искали. Следуйте за мной.

4.2.1). по 1.1

4.2.2). по 1.2

\section{Ласковый убийца}

[диалог с Керри]

- Ну-с, молодой человек? Вы меня искали - вы меня нашли. Что дальше?

- Понятия не имею. Отец в письме сказал, что вы сообщите все необходимое.

- Значит, Ланак все же решил действовать через меня... Интересно... Неужели наконец-то стал доверять? Так, ладно. Сам-то мне ничего не хочешь сказать?

1). - Вас нелегко было найти. Кто вы...ты такой?

- Я? Хех… Всего лишь скромный глава милой организации под названием гильдия убийц.

1.1). - Убийц? Но...это же незаконно?

- Официально, разумеется, наша деятельность вне закона. Но новый император понял, что проще контролировать хорошо отлаженную организацию, чем бороться с разрозненными неорганизованными структурами. Таким образом, сейчас наша деятельность жестко ограничена, но императорская тайная стража - а это те еще бойцы, я тебе скажу - не препятствует нашей работе, покуда она не противоречит планам императора.

1.1.1.1). - Убийство узаконено? И самим Императором, о доброте и справедливости которого трубят на каждом углу? Мерзость какая.

- Не спеши с выводами. Я же уже сказал, что наши действия полностью согласуются с людьми Императора. Клиент называет имя - и доверенные люди Императора узнают всю подноготную этого человека. Если он чист перед законом - мы сообщаем клиенту о невозможности выполнить работу. Но чаще всего заказы приносят на самую мразь. Причем приносят их тоже не обремененные моралью господа. Таким образом, мы лишь помогаем Императору убирать сорняки с донельзя заросшей пашни.

1.1.1.1.1). по 1.3.1

1.1.1.1.2).по 1.3.2

1.1.1.2). - Убийства разрешены? Весьма удобно...

- Не спеши с выводами. Я же уже сказал, что наши действия полностью согласуются с людьми Императора. Клиент называет имя - и доверенные люди Императора узнают всю подноготную этого человека. Если он чист перед законом - мы сообщаем клиенту о невозможности выполнить работу. Но чаще всего заказы приносят на самую мразь. Причем приносят их тоже не обремененные моралью господа. Таким образом, мы лишь помогаем Императору убирать сорняки с донельзя заросшей пашни.

1.1.1.2.1). по 1.3.1

1.1.1.2.2). по 1.3.2

1.2). –Хм… Значит, я не сильно ошибся в своих предположениях.

- Люблю неглупых людей.

- Польщен. Но что мне делать дальше?

1.3). – Благие… Но как мой отец связан с грязными убийцами?

- Прямо так уж грязными, ужасными и противными? Забываешь, с кем говоришь, мальчик. Но даже не в этом дело. Не спеши с выводами. Наши действия полностью согласуются с людьми Императора. Клиент называет имя - и доверенные люди Императора узнают всю подноготную этого человека. Если он чист перед законом - мы сообщаем клиенту о невозможности выполнить работу. Но чаще всего заказы приносят на самую мразь. Причем приносят их тоже не обремененные моралью господа. Таким образом, мы скорее подобие тайной стражи, чем беспринципные грабители.

1.3.1). - Хм… Что ж, похоже, я поторопился с выводами. Что мне делать дальше?

1.3.2). – Красивая сказка, но я в нее не верю.

- Жаль. Впрочем, это ничего не меняет. Ты должен понимать, что отец не стал бы доверять уличному разбойнику.

1.3.2.1). - Согласен…

1.3.2.2). – Его тоже можно обмануть.

- Это тупиковый разговор. Так или иначе, я единственный знакомый тебе человек, знающий кое-что об истинной жизни твоего отца.

1.4). – Почему вас до сих пор не перебила стража?

- Официально, разумеется, наша деятельность вне закона. Но новый император понял, что проще контролировать хорошо отлаженную организацию, чем бороться с разрозненными неорганизованными структурами. Таким образом, сейчас наша деятельность жестко ограничена, но императорская тайная стража - а это те еще бойцы, я тебе скажу - не препятствует нашей работе, покуда она не противоречит планам императора. Наши действия полностью согласуются с людьми Императора. Клиент называет имя - и доверенные люди Императора узнают всю подноготную этого человека. Если он чист перед законом - мы сообщаем клиенту о невозможности выполнить работу. Но чаще всего заказы приносят на самую мразь. Причем приносят их тоже не обремененные моралью господа. Таким образом, мы скорее подобие тайной стражи, чем беспринципные грабители.

1.4.1). – Ясно. Что мне делать дальше?

1.4.2). по 1.3.2

2). - А дальше ваша очередь, господин Керри. Где мне найти отца и что вообще происходит?

- Ага... Значит ты - сын Ланака? Неплох, неплох. Для начала мне, наверное, стоит представиться. Я + по ветке 1 начиная со слов скромный глава...

[при любом течении диалога следующая реплика Керри]

- Что касается твоего отца...

- Да?

- Он передал мне, что вскоре придёт некий юноша, и попросил помочь ему - то есть, тебе - обжиться в городе, научить кое-каким полезным вещам...

1). - Мне надоело плясать под чужую дудку. Где найти отца?

- Понятия не имею. Он сказал, что встретится с тобой. Позже, когда уладит свои дела.

1.1). - Проклятый его побери... Что ж, выбора у меня нет. Чем предлагаешь заняться?

- Поможешь мне в нескольких делах, заодно кое-чему научишься, да и деньжат подзаработаешь.

1.1.1). по 3.1

1.1.2). по 3.2

1.2 ). - А хотя бы кто он такой?

- Думаю, это пока не важно. Могу сказать одно - он помогал мне, когда я только начинал создавать нашу организацию. Потом, правда, наши дороги разошлись. Сейчас вот опять...

[возврат на 1.1]

2). - Ну что ж, проведу время с пользой. А там, глядишь, и отец пожалует. Чем предлагаешь заняться?

- Поможешь мне в нескольких делах, заодно кое-чему научишься, да и деньжат подзаработаешь.

2.1). по 3.1

2.2). по 3.2

3). - Полезным вещам? Например?

- Да всему, что сам умею. Ну, не бесплатно, разумеется. Впрочем, достать звонкие золотые кругляшки я тебе помогу.

3.1). - Отлично.

3.2). - Не собираюсь делать за тебя твою работу.

- В городе без денег ты долго не протянешь. К тому же Ланак сам просил меня занять тебя чем-нибудь, чтоб дурью не маялся.

- Хорошо, согласен.

4). - А кто он хотя бы такой? 

-  Думаю, это пока не важно. Могу сказать одно - он помогал мне, когда я только начинал создавать нашу организацию. Потом, правда, наши дороги разошлись. Сейчас вот опять...

- Ладно. Что мне делать сейчас?

- Ждать. Но в городе без звонкой монеты ты долго не протянешь, да и работу сам сразу не найдешь. Посему, поможешь мне в нескольких делах, подзаработаешь деньжат, да и научишься кое-чему полезному...

4.1). по 3.1

4.2). по 3.2

\chapter{Тяжёлый труд}

\section{Да, господин Керри}

[при любом течении диалога]

- Прежде всего тебе необходимо усвоить некоторые навыки, без которых в нашем деле обходиться весьма непросто. Согласен, ты неплохо умеешь размахивать железом. Но искусство боя владением мечом отнюдь не ограничивается. Тем более, что и мечом ты сражаешься несколько...э-э...топорно и примитивно. 

1). - Быть может...

- Я могу научить тебя быть более выносливым, лучше обращаться с легким оружием, пользоваться всеми видами стрелкового. Кроме того, нелишним будет умение взламывать замки и быстро бегать.

1.1) Пожалуй, в первую очередь стоит научиться искусному бою на клинках.

- Полностью согласен с твоим выбором. Легкое холодное оружие - рапиры, шпаги, кинжалы, стилеты - отличный инструмент в случае, когда надо без лишнего шума расправиться с плохо защищенным противником. Но для грамотного обращения с таким оружием нужна немалая ловкость и сноровка.

- Приступим к обучению?

- Разумеется. Итак, запоминай. Самое главное при сражении на шпагах и рапирах - правильно перемещаться и уходить от атак. Помни, что главное твое преимущество - это скорость движений и малый вес оружия, вследствие чего противник почти наверняка не сможет парировать твою атаку, а тем более уклониться от нее. Но и твоя защита против, скажем, двуручного меча будет весьма проблематичной... Поэтому движение, движение и еще раз движение. Кроме того, надо научиться находить незащищенные точки противника - открытые участки тела, сочленения доспехов и тому подобное. Для начала этого должно хватить.

- Спасибо.

- Пока не за что. Теория ничто без практики. Так что теперь проверим, на что ты способен. Недалеко от южных ворот города обосновались какие-то болваны в доспехах южных наемников. Все бы ничего, но они начали грабить проходящих мимо путников. Стражники уже хотели разобраться, но тут я узнал о тебе и попросил повременить. Так что они и будет твоим испытанием. Убей их любым легким оружием. Да, кстати, доспехи у них отличные. И учти, там, так же как и везде неподалеку от города есть мои информаторы. Поэтому не пытайся схитрить и перестрелять их из арбалета. Помни, что это своеобразный экзамен.

1.2) - Думаю, мне не помешает потренировать выносливость.

- ...

1.3) - Умение стрелять никогда не будет лишним.

- А особенно в нашем деле. Пусть все эти рыцари, дворяне что угодно говорят о честном поединке... По их мнению, лук - оружие неблагородное. Но нам плевать. Важно не средство, а результат. Для тихого убийства нет ничего лучше обычного арбалета или шнеппера. А лук сгодится для охоты и...

- Может, начнем уже?

- Да, разумеется. Для точной стрельбы нужен хороший глазомер и внимательность. Важно помнить про ветер, который может изменить даже траекторию полета болта, не говоря уже о стрелах. Старайся целиться не слишком долго, иначе руки могут устать и дрогнуть в неподходящий момент. Думаю, пока этого достаточно.

- Это все?

- Разумеется, нет. Теперь попробуй применить эти знания. У меня есть заказ на некоего Пьера Кедисона, купчишки из торговой гильдии. Он пытается наладить регулярный контрабандный вывоз оружия, наживаясь за счет казны. Но убить его надо так, чтобы никто не узнал обстоятельств смерти, что в городе почти невозможно.

- И что же ты предлагаешь?

- Я бы на твоем месте послал ему письмо от имени друга с просьбой встретиться где-нибудь в безлюдном месте. И подождать его там. Загвоздка лишь в подписи этого самого друга, без которой купчишка может что-то заподозрить. Поэтому подпись должна быть, причем настоящая. Как ты этого добьешься - твое дело.

1.3.1). - Мда... Действительно, не слишком-то просто. А другие варианты есть?

- Не уверен, что он проще...

- Я слушаю.

- Можешь попробовать тихо и незаметно убить стажу около его дома, а затем и его самого, когда он выйдет. Однако поверь, стражники не лыком шиты, и если тебя заметят, то за твою жизнь не дам и гроша...

- Я подумаю.

- Да, это полезно. Возможно, есть и другие способы. Важно одно - отсутствие свидетелей. Кстати, один торговец недавно расхваливал мне какие-то орочьи капканы и ловушки, но я всей этой их техники не очень-то доверяю...  

1.3.2). - Слушай, ты сказал, я должен стрелять, а не писать подложные письма и бегать за жертвой.

- Ну, не всегда все просто получается. Во-первых, это будет полезный опыт для тебя. Во-вторых, надо же и головой думать иногда. Поверь мне, просто хороший стрелок обычно живет раза в три меньше, чем умный хороший стрелок.

- Хорошо. Но нет способа попроще?

- Можешь попробовать тихо и незаметно убить стажу около его дома, а затем и его самого, когда он выйдет. Однако поверь, стражники не лыком шиты, и если тебя заметят, то за твою жизнь не дам и гроша...

- Я подумаю.

- Да, это полезно. Возможно, есть и другие способы. Важно одно - отсутствие свидетелей. Кстати, один торговец недавно расхваливал мне какие-то орочьи капканы и ловушки, но я всей этой их технике не очень-то доверяю...

1.4) Меня прельщает возможность взламывать замки.

- Это весьма полезное умение для нашего дела. Но, будучи в городе, не забывай о разумной осторожности - стражники не будут в восторге, узнав, что кто-то шастает по чужим домам. А прохожие, заметив тебя копающимся у чужой двери, не преминут сообщить страже или даже нападут сами.

- Понятно.

- Теперь слушай. Для грамотного обращения с отмычками нужно "чувствовать" замок. Всегда анализируй, какой тип замка перед тобой и какими отмычками следует действовать. Старайся не нажимать слишком сильно на движущиеся части механизма - очень велика вероятность сломать замок.

- Хорошо.

- Проверим, насколько быстро и аккуратно ты справишься со взломом в реальных условиях. Два дня назад тайная стажа поручила нам обыскать дом городского судьи - есть подозрение, что результат последнего процесса был сформирован искусственно. Тайные решили провести неофициальный обыск и, разумеется, поручили это дело нам... Ну а я - тебе. Принесешь мне все бумаги, которые сможешь найти. Думаю, мне не надо говорить, что никто не должен тебя там заметить.

[прохождение до безобразия банально. Игрок идет к дому, дожидается момента, когда рядом никто не будет проходить, вычищает все сундуки. Или же берет только письмо, нужное для квеста. Затем отдает его Керри. Кстати, его можно прочитать]

Текст письма:

- Делай что хочешь, но Маккера должны казнить. После суда уничтожь все документы по процессу. Как ты это сделаешь для нас несущественно. Сделаешь все правильно - в доме наемников Фред передаст тебе вознаграждение. Нет - ну ты понимаешь...

[диалог с Керри]

- Вот письмо. Больше ничего интересного я там не нашел.

- Этого достаточно. Ты отлично выполнил свою работу.

1.5) - Хотелось бы научиться бегать быстрее.

- ...

2). - Ну-ну... Проверим?

- Ха-ха. Полегче, не горячись. Я отнюдь не хотел тебя оскорблять. Да и рановато тебе со мной тягаться.

- Так-то лучше.

- Итак, чем хочешь заняться?+ по первой ветке, начиная со слов "Я могу научить тебя..."

3). - Думаю, я уже умею немало всего. Ты говорил о каких-то заданиях...

- Да. Но займешься ты другим делом. В последнее время мне сообщают, что святоши через третьих лиц покупают у орочьих купцов оружие. А применение его, как известно, только одно. Боюсь, готовится бунт, а посему узнай-ка, что там у них творится и кто всем заправляет.

3.1). - Хорошо, попробую.

3.2). - Э-э...как?

- Благие, да как угодно. Пора тебе подумать своей головой.

\section{Нет, господа бандиты}

[диалог с этими челами]

1). - Привет. Я смотрю, вы тут скучаете?

- Уже нет. Развлечение пришло.

- Предлагаю вам развлечься. Но всем должно быть интересно, посему нападайте-ка по-одному.

- Ха-ха. Идет. Но тебе это не поможет.

[бой, противники, за исключением двух последних, нападают по-одному]

2). - Ничего личного, но мне придется вас убить. Тренировка и все в этом роде

- Ха-ха... Приятель, да ты был юмористом.

[бой]

Уже у Керри.

- Готово.

- Ну что ж, поздравляю. Теперь ты можешь использовать полученное умение.

\section{Нет, господин торговец}

[Варианты прохождения квеста]

1. Самый тупой.

Игрок выносит стражу и убивает целевого перса. Это должно быть очень сложно, стражники стоят так, что видят друг друга и чтобы найти, с кого начать, надо немало раз перезапустится (если игрока замечают, его убивают без вариантов, если игрок стреляет в стражника, то в этой миссии он сразу умирает)

2. Более интересный, но до него надо догадаться самому. Целевой чел ходит по определенному маршруту туда-сюда, скажем, от дома до гильдии торговцев. Игрок замечает место, где он проходит, и в отсутствие стражи, которая ходит вместе с целью, ставит ловушку. Кстати, стоить она должна очень недешево.

3. Самый умный. Но и самый сложный. Игрок идет к обозначенному на карте другу целевого чела. Диалог с ним:

- Приветствую. Чем обязан?

1). - Вас рекомендовали мне как выдающегося торговца редкостями... Хотелось бы заключить с вами договор.

- На каких условиях?

- О, все просто. Вы привозите мне дюжину орочьих ловушек и я плачу 300 золотых.

- 1800, половину суммы сейчас, половину после доставки.

- Идет. Но в таком случае, заключим договор. Вот бумаги. Подпишите внизу.

- Но половина листов не заполнена?

- [лидерство или торговля] Это прослойка чтобы чернила не пачкали станицы. Их тоже надо подписать, это новое слово в современном делопроизводстве, полностью исключающее возможность подделки.

- Странно... Ну что ж, вот. Товар будет через 9 дней.

- До встречи, любезнейший.

2).[харизма] - Вот, хочу познакомиться со столь выдающимся торговцем...

- Меня зовут Рей. Рей Браден, к вашим услугам.

- Выпьем за знакомство?

- Разумеется.

(через несколько часов)

- Эт-то...Распишись здесь. Да нет, снизу слева. Ну, которое справа.

- Зач...зачем?

- Ну мы друзья? Вот и подпиши.

- Друзья. Да, это... Вот.

- Мне пора. 

3).[сила] - Ни звука или умрешь. Подпиши снизу эту бумагу и останешься жив.

- Еще чего! Охрана! Охрана!

[бой. Победит герой или нет, но задание провалено и выполнять его придется другим путем]

После первых двух вариантов игрок получает письмо с подписью и относит сам целевому персонажу, после чего приходит на поляну за городом и спокойно убивает его.

- Готово. Должен сказать, это было нелегко.

- Ты неплохо справился и теперь умеешь стрелять. Да, вот тебе небольшая награда за выполнение задания.

\section{Не знаю, господа священники!}

[структура выполнения задания]

1. Посещение церкви.

1.1. Диалог со священником

---

- Святой отец, позвольте вас отвлечь...

- Да?

1). - Ваша проповедь невероятно убедительна. Но... Зачем настраивать прихожан против магов, и их деяний?

- Да убей меня, если я понимаю. Первосвященник и иже с ним в последнее время совсем взбеленились. А против их воли не пойдешь.

1.1). - То есть, вам это тоже не нравится?

- Как сказать... Мне многое не нравится в последнее время. Но обсуждать это с тобой мне некогда. Ступай. [возврат на 1 и потом 1.2]

1.2). - И где мне найти этих самых, которые с ним?

- Первосвященник, знамо дело, в своем доме за церковью. А приближенные его - кто знает.

1.2.1). - Ясно.

1.2.2). - Приближенные? Ты его, случаем, с императором не путаешь, старик?

- Похоже, немного ты о нашем первом знаешь... 

2). - Меня прислал...ну, вы понимаете, кто меня прислал. Когда начинаем?

- Совершенно не понимаю. Ты меня с кем-путаешь. Если к наемникам нашим послали, то вон они, в конце площади стоят.

2.1). - Наемники? Зачем они вам?

- Первосвященник собирает. К нему и вопросы.

- Вот как... А где его искать?

- У него дом за церковью, там спроси.

2.2). - Хорошо, благодарю.

3). Скажите, а не может ли молодой человек, неплохо владеющий мечом, найти у вас работу?

- У нас, если ты не собираешься читать проповеди, нет. А так, вон на площади собрались эти обалдуи, которых Первосвященник набирает. Пойди, спроси. 

---

1.2. Диалог с двумя наемниками, стоящими на площади около церкви.

- Здорово.

- Ага. Что хотел?

1). - Я впервые в городе. Где тут можно заработать?

- Хм... Это смотря что ты умеешь...

- Ну, сражаюсь неплохо.

- Тогда советую к Первосвященнику зайти. Он набирает людей, платит вдвое против обычного. И работенка пока не пыльная.

2). Что тут торчите, вид портите?

- Вали отсюда, некогда с тобой связываться.

---

2). Посещение торговых рядов орков.

Диалог с торговцами.

- Что желаете?

1). - Меня интересует один вопрос... Кто скупает у вас оружие в последнее время? Причем, в весьма большом количестве.

- Любые сведения - тот же товар. Тем более, что ценный.

- И?

- 2000 золотом.

- Ого... А не много будет?

- Много. Ну так что?

1.1). - Идет.

1.2). - Попробую решить свои проблемы другим способом...

- Эй, стой. Тысяча.

1.2.1). - Вот теперь другое дело.

1.2.2). - Нет, изволь набить карман за счет кого-нибудь другого.

2). - Разумеется, посмотреть твой товар.

---

Далее игрок либо идет к этому самому первосвященнику, либо продолжение диалога с торговцем:

- Итак, что ты знаешь?

- Пару недель назад ко мне зашел незнакомый человек и договорился о регулярных поставках оружия и доспехов. А я, не будучи дураком и понимая, что им впоследствии почти наверняка заинтересуются тайные, а может, и еще кто, отправил подмастерье проследить за ним, выяснить, где он живет. И ... остался без хорошего работника. Вскоре этот человек вновь посетил меня с аналогичным предложением. Сейчас он захаживает через каждые два-три дня и А следующий раз должен быть у меня послезавтра. Вот и все что мне известно.

- Хм... Негусто. Ну да ладно. Через день, говоришь?..

[далее игрок следит за этим челом, по дороге убивает несколько охранников, следующих за целевым персом на почтительном расстоянии. И оказывается у дома первосвященника]. Здесь ветка квеста сливается с первой. Соответственно, следующая цель - взломать замок и войти внутрь. Там должен быть нпс, за которым шел игрок. Или не шел. Все равно пусть будет :-) . Дальше игрок шарится по дому, находит некий Важный Документ и уходит.

[диалог с Керри]

- Вот бумаги, которые я нашел в доме первосвященника.

- (читает) Все даже хуже, чем я думал...

1). - И?

- И значит, поспать я теперь смогу не скоро.

- А что делать мне?

- Все, что угодно. Отдохни, сходи в трактир, погуляй по городу... Или же зайди в холл гильдии и поздоровайся с отцом, который ждет тебя там уже пару часов.

2). - А, и мне, разумеется, надо в очередной раз бегать по чужим домам, убивать и грабить людей?..

- Не совсем. Теперь надо найти на улицах города гонца в заляпанных вчерашним соусом штанах... Ха-ха-ха. Да шучу я. Займись чем хочешь.Отдохни, сходи в трактир, погуляй по городу... Или же зайди в холл гильдии и поздоровайся с отцом, который ждет тебя там уже пару часов.

3). - И в чем суть заговора? Что вообще происходит?

- Если вкратце, то святым отцам опять захотелось власти. Причем в этот раз на их стороне гораздо больше людей. Расслабились тайные, сильно расслабились. Ну да это уже наша головная боль, никак не твоя.

3.1). по ветке 1

3.2). по ветке 2

\section{Ага, отец}

- Ну наконец-то дождался. Хотел уже поисковую связку разворачивать.

1). - Здравствуй, батя. И что все это значит?

- Меня всегда раздражали вопросы, на которые нельзя ответить парой слов. Эх... Начать, наверное, надо с того, что я не только хороший кузнец и просто веселый мужик. Я еще и доверенное лицо императора, его… советник, так скажем. Так-то, сын.

1.1). - Советник Императора - в деревне?

- Помимо прочих моих талантов, я еще и маг, хотя весьма посредственный. Так что расстояние - не такая большая проблема. Да и плюсов в удаленности от столицы и лишних глаз, ушей и прочих… частей тела немало. Но речь сейчас не о том.

1.2). - Хм... Весело. А где ты был последнюю неделю? Мог бы и заранее предупредить.

- Как бы так половчее завернуть... Срочные дела государственной важности принудили меня срочно обратить на них пристальное внимание.

2). - Рад видеть. Ты где пропадал?

- Как бы так половчее завернуть... Срочные дела государственной важности принудили меня обратить на них пристальное внимание.

- Ты и государственные дела??

- Эх... Начать, наверное, надо с того, что я не только хороший кузнец и просто веселый мужик. Я еще и доверенное лицо императора, его… советник, так скажем. Так-то, сын.

2.1). по 1.1

2.2). по 1.2

3). - Связку? Ты о чем?

- И разумеется, в магии ты до сих пор ни бум-бум... Надо было раньше наверное этим заняться... Связки – это плетения. Хотя что ты поймешь так сразу... Магия это ... в общем, сейчас не о том речь.

- Согласен.+по ветке 2, начиная со слов "Ты где пропадал?"

[при любом ведении диалога]

- Ладно. Что дальше?

- Даже не знаю, что тебе сказать. Вроде много всего надо... В последнее время немало ерунды происходит. Большинство - мелочь, вроде церковников, которые даже не представляют, против каких сил пытаются выступить. С ними вполне успешно разбираются наши тайные. Но есть и другие проблемы. И тут-то я и подумал, что хватит уже моему оболтусу бездарно убивать время в глуши.

- И в чем будет заключаться моя неоценимая помощь?

- Пока не знаю. А чтобы тебе не тратить зря свое драгоценное время... Загляни-ка в гильдию магов, тебе пригодятся кое-какие знания по части вероятности. И ну-ка лицо повеселее. Вот тебе пара тысяч на карманные расходы.

1). - Не то чтоб я всю жизнь мечтал слушать скучные нотации этих стариков-академиков... Ну да ладно.

2). - Мне вполне хватает меча и прочих полезных железок. И нет никакого желания слушать этих брюзжащих профессоров, или как там они себя называют?

- Раньше ты не прекословил... Все твои умения по части махания острыми предметами ничего не стоят по сравнению с моими. Магия дает очень большие преимущества. Могу показать. Хотя нет, а то еще убью ненароком. В общем, отправляйся в гильдию. Или я тебя сам отправлю, причем одновременно на все четыре стороны.

- Хех, батя, твое чувство юмора все так же хромает. Уже иду, не кипятись.

(задание – изучить хотя бы одну стихию)

\section{Угу, маги}

- Здра...

- Что надо?

- Меня к вам попросил зайти отец...

- Проваливай, и передай, что он идиот.

1). - Моего отца зовут Истрен. Истрен Толь. Так что, передать?

2). - Полегче, кмет. Я не деревенский увалень, и не привык, когда со мной так разговаривают.

- Что тебе нужно? Мы не обучаем сторонних людей. Только детей членов гильдии.

- Мой отец - Истрен Толь. Думаю, больше вопросов нет?

- Ээ...да... Пройдите в холл и поговорите с главой нашей гильдии. Это его дело, и я не могу ничего указывать.

[...]

- Добрый день. Я...

- Да-да, рад что Вы почтили нас своим присутствием. Желаете узнать что-либо, или же не будем тратить время и сразу же займемся обучением?

1). - Я хотел бы изучить основы стихийной магии.

- Чем именно займемся? Учти, обучение не будет легким.

1.1). - Меня интересует изучение заклинаний магии огня.

- Хорошо. Прежде всего, забудь про эти варварские слова - магия, заклинания... Это все для простецов. Все, что мы делаем – управляем некой материей, для большинства невидимой, что вполне объяснимо с научной точки… ну да мы попусту тратим драгоценное время. Дар у тебя есть, но нужно уметь с ним управляться.

- Ясно. Может, начнем?

- Да, разумеется. Сейчас заложу в тебя первоначальные навыки. Не дергайся, это что-то навроде гипноза уличных фокусников.

...

- Ну вот, собственно, и все. Теперь ты имеешь представление о принципах обращения со стихией огня.

- Это все?

- Разумеется, нет. Практика, практика, молодой человек... Теперь небольшое испытание. Так…эмм... что бы тебе такого придумать... небольшой лабиринт с искусственно созданными агрессивными жизненными формами.

- То есть меня сейчас опять будут убивать?

- Это всего лишь воображаемые враги, так что умереть ты не умрешь. Но испытание придется начать сначала. Удачи. Кстати, если повезет, сможешь найти несколько полезных амулетов. Они настоящие и останутся у тебя после...хм...урока.

[...]

1.2). - Меня интересует изучение заклинаний магии воды.

- ...

- Ну вот, собственно, и все. Теперь ты имеешь представление о принципах обращения со стихией земли. И, дабы не изобретать колесо… (перс опять же попадает в какое-нибудь мясное место…)

1.3). - Меня интересует изучение заклинаний магии земли.

- ...

- Ну вот, собственно, и все. Теперь ты имеешь представление о принципах обращения со стихией земли. И, дабы не изобретать колесо… (перс опять же попадает в какое-нибудь мясное место…)

2). - Хотелось бы приобрести себе некоторые вещи...

- Предупреждаю, твой кошелек сильно похудеет.

3). - Хотелось бы узнать о эффективности магии различных стихий.

- Эффективности? А здесь нет и не может быть однозначных категорий. Все зависит от твоих склонностей, способностей, конкретной ситуации и целей… Правда, важно понимать, что наилучшие результаты при должном опыте приносит комбинирование различных стихий и эффектов.

4). - А не нужна ли вам какая-нибудь помощь?

- Помощь…хмм… Мне пожалуй что нет, но уверен, кое у кого из здешних низших магов и учеников должны найтись кое-какие поручения. Если, конечно, тебя интересует такой способ заработка.

\section{Я гарантирую это!}

1). - Я изучил магию стихий и уже кое-что умею.

- Да почти что ничего ты почти не умеешь. И опыта ноль. Ну да ладно, начало положено. Теперь садись, побеседуем.

1.1). – Весь внимание.

- Так, с чего бы начать… В общем, ерундовых проблем сейчас у меня немало, но то по большей части рядовые, несплоченные кучки интриганов или же склоки старых брюзжащих святош. Немного волнует другое. Уже ставшие мифом, детской сказкой эльфы. Думаю, ты в курсе, что несмотря на многовековую тишину, небольшие отряды охраняют выходы из Великого леса, проклятый его побери… И чуть больше недели назад выяснилось, что два отряда, стоявшие у южных склонов, бесследно исчезли. В лагерях их никаких следов драк. Вообще никаких следов. Оставлена недоеденная похлебка, костяшки домино на столах. А людей нет. Если бы там кровь, трупы – то ладно, может, разбойники. А так…тревожно… И…еще кое-что есть. В одном из немногих имеющихся у нас свитке, написанном эльфами еще в годы мира, были слова. «И будет война, ужасная, и жертв не счесть. Мы уйдем в глушь, в родную глушь, в леса. Пройдут века. VII веков. И мы вернемся, дабы уничтожить ненасытное, бессмысленно кровожадное племя». Так вот, у остроухих много всякого бреда понаписано. Но вспомнил вот, перечел. Так вот, как раз 701 год, считая по их календарю, от нашей победы. И… тревожно. Но это лишь древнее пустословие, ничем не подтвержденное. А вот исчезновение полусотни вооруженных и хорошо обученных человек похуже будет. Как знать, вдруг и впрямь война. А посему, думаю отправить послов к оркам, с которыми давно уже нет никаких контактов, окромя торговых. Пока никакой конкретики на военную тему, просто заверение дружеских отношений и т.д. и т.п. Так вот, пора тебе уже вникать в государственные дела, моим преемником на этой должности будешь, надеюсь. Так что отправляйся с нашими послами, дипломатии поучишься, может, себя как проявишь. А там поглядим.

1.2). – И?

- Не торопись, да и меня не торопи. Так, с чего бы начать… В общем, ерундовых проблем сейчас у меня немало, но то по большей части рядовые, несплоченные кучки интриганов или же склоки старых брюзжащих святош. Немного волнует другое. Уже ставшие мифом, детской сказкой эльфы. Думаю, ты в курсе, что несмотря на многовековую тишину, небольшие отряды охраняют выходы из Великого леса, проклятый его побери… И чуть больше недели назад выяснилось, что два отряда, стоявшие у южных склонов, бесследно исчезли. В лагерях их никаких следов драк. Вообще никаких следов. Оставлена недоеденная похлебка, костяшки домино на столах. А людей нет. Если бы там кровь, трупы – то ладно, может, разбойники. А так…тревожно… И…еще кое-что есть. В одном из немногих имеющихся у нас свитке, написанном эльфами еще в годы мира, были слова. «И будет война, ужасная, и жертв не счесть. Мы уйдем в глушь, в родную глушь, в леса. Пройдут века. VII веков. И мы вернемся, дабы уничтожить ненасытное, бессмысленно кровожадное племя». Так вот, у остроухих много всякого бреда понаписано. Но вспомнил вот, перечел. Так вот, как раз 701 год, считая по их календарю, от нашей победы. И… тревожно. Но это лишь древнее пустословие, ничем не подтвержденное. А вот исчезновение полусотни вооруженных и хорошо обученных человек похуже будет. Как знать, вдруг и впрямь война. А посему, думаю отправить послов к оркам, с которыми давно уже нет никаких контактов, окромя торговых. Пока никакой конкретики на военную тему, просто заверение дружеских отношений и т.д. и т.п. Так вот, пора тебе уже вникать в государственные дела, моим преемником на этой должности будешь, надеюсь. Так что отправляйся с нашими послами, дипломатии поучишься, может, себя как проявишь. А там поглядим.

2). – Что дальше?

- Дальше… Садись, побеседуем. Так, с чего бы начать… В общем, ерундовых проблем сейчас у меня немало, но то по большей части рядовые, несплоченные кучки интриганов или же склоки старых брюзжащих святош. Немного волнует другое. Уже ставшие мифом, детской сказкой эльфы. Думаю, ты в курсе, что несмотря на многовековую тишину, небольшие отряды охраняют выходы из Великого леса, проклятый его побери… И чуть больше недели назад выяснилось, что два отряда, стоявшие у южных склонов, бесследно исчезли. В лагерях их никаких следов драк. Вообще никаких следов. Оставлена недоеденная похлебка, костяшки домино на столах. А людей нет. Если бы там кровь, трупы – то ладно, может, разбойники. А так…тревожно… И…еще кое-что есть. В одном из немногих имеющихся у нас свитке, написанном эльфами еще в годы мира, были слова. «И будет война, ужасная, и жертв не счесть. Мы уйдем в глушь, в родную глушь, в леса. Пройдут века. VII веков. И мы вернемся, дабы уничтожить ненасытное, бессмысленно кровожадное племя». Так вот, у остроухих много всякого бреда понаписано. Но вспомнил вот, перечел. Так вот, как раз 701 год, считая по их календарю, от нашей победы. И… тревожно. Но это лишь древнее пустословие, ничем не подтвержденное. А вот исчезновение полусотни вооруженных и хорошо обученных человек похуже будет. Как знать, вдруг и впрямь война. А посему, думаю отправить послов к оркам, с которыми давно уже нет никаких контактов, окромя торговых. Пока никакой конкретики на военную тему, просто заверение дружеских отношений и т.д. и т.п. Так вот, пора тебе уже вникать в государственные дела, моим преемником на этой должности будешь, надеюсь. Так что отправляйся с нашими послами, дипломатии поучишься, может, себя как проявишь. А там поглядим.

3). – Теперь я не хуже заправского уличного паяца-фокусника. И зачем это было надо?

- Пригодится, и не единожды. Так, теперь садись, побеседуем. Так, с чего бы начать… В общем, ерундовых проблем сейчас у меня немало, но то по большей части рядовые, несплоченные кучки интриганов или же склоки старых брюзжащих святош. Немного волнует другое. Уже ставшие мифом, детской сказкой эльфы. Думаю, ты в курсе, что несмотря на многовековую тишину, небольшие отряды охраняют выходы из Великого леса, проклятый его побери… И чуть больше недели назад выяснилось, что два отряда, стоявшие у южных склонов, бесследно исчезли. В лагерях их никаких следов драк. Вообще никаких следов. Оставлена недоеденная похлебка, костяшки домино на столах. А людей нет. Если бы там кровь, трупы – то ладно, может, разбойники. А так…тревожно… И…еще кое-что есть. В одном из немногих имеющихся у нас свитке, написанном эльфами еще в годы мира, были слова. «И будет война, ужасная, и жертв не счесть. Мы уйдем в глушь, в родную глушь, в леса. Пройдут века. VII веков. И мы вернемся, дабы уничтожить ненасытное, бессмысленно кровожадное племя». Так вот, у остроухих много всякого бреда понаписано. Но вспомнил вот, перечел. Так вот, как раз 701 год, считая по их календарю, от нашей победы. И… тревожно. Но это лишь древнее пустословие, ничем не подтвержденное. А вот исчезновение полусотни вооруженных и хорошо обученных человек похуже будет. Как знать, вдруг и впрямь война. А посему, думаю отправить послов к оркам, с которыми давно уже нет никаких контактов, окромя торговых. Пока никакой конкретики на военную тему, просто заверение дружеских отношений и т.д. и т.п. Так вот, пора тебе уже вникать в государственные дела, моим преемником на этой должности будешь, надеюсь. Так что отправляйся с нашими послами, дипломатии поучишься, может, себя как проявишь. А там поглядим.

(реплики на любой из вариантов)

1). – Отлично, интересно будет посмотреть, как там у этих зеленокожих образин.

- Ты чем меня слушал, оболтус? Ты теперь посол. Так что дипломатия, вежливость, и ничего окромя! «как поживают наши славные союзники» ты хотел сказать.

2). – Неплохо, все интереснее, чем тут болтаться.

3). – Отлично.

4). – С умным лицом сидеть на светских приемах этих идиотов?

- Да, постарайся хотя бы научиться делать умное лицо. А что касается приемов и великосветской мишуры…не волнуйся, там все проще. Ну да сам увидишь.

- Когда отправляемся?

- Если готов, то через пару часов. Вы отправляетесь. Я, разумеется, остаюсь.

1). – Эээ…а что мне там делать?

- А это сам решай. Пора уже как-то себя показать. Цель тебе известна, добиться расположения орков и их готовности в случае чего оказать военную помощь.

2). – До встречи. Я готов, в путь.

3). – У меня есть здесь еще кое-какие дела. Я скоро буду.

\chapter{Ещё более долгая дорога}

\section{Знакомство}

Диалог сразу же после выезда из города.

- Так-с, ну здравствуй. Меня зовут Бондри Слей. И я, а не кто-нибудь другой руководит этой...эмм...экспедицией. Хочу сразу...

1). -[перебить] А мне что за дело до того? Пошел прочь, я занят.

- Ты...ты как разговариваешь со мной? Здесь я принимаю решения и отдаю приказы. И плевать, кто там о тебе хлопотал, будь ты хоть наследником Императора - только я указываю, что, как и кому делать. Посему, повежливее, олух.

1.1). - Самодур брехливый... Терпеть не могу идиотов. Защищайся.

- Ты с ума сошел? От этого путешествия, быть может, зависит существование империи, я не могу допустить драки между нашими людьми, а потому не приму твой вызов. До нескорой, надеюсь, встречи.

1.2). - Император назначил тебя координатором экспедиции? По тону ты склочный стряпчий или трактирщик из какого-нибудь притона, никак не иначе.

- Хвост проклятого... Я...ээ... Лорд Крондер неожиданно скончался и меня поставили на его место. И что это я с тобой разглагольствую... Придержи язык, олух, и впредь повежливее.

1.3). - О, какой пассаж, какая экспрессия, какие гневные ноты... Похлопал бы, да лень руки поднимать. И что сообщить великодушно изволите?

- Эээ...чего? Я...в общем, это, выполняй все мои указания и без моих слов ни шагу.

1.3.1). по 1.1

1.3.2). по 1.2

1.3.3). Идет, не кипятись. 

2). -[перебить] Меня...

- Плевать, как там тебя папаша называет. От тебя требуется лишь одно - заруби себе на носу, что принимаю решения и отдаю приказы здесь я. И плевать, кто там о тебе хлопотал, будь ты хоть наследником Императора - здесь я указываю, что, как и кому делать.

2.1). по 1.1

2.2). по 1.2

2.3). по 1.3

3). -[слушать дальше] ...предупредить. Никаких самостоятельных действий без моих прямых на то указаний. Все ясно?

3.1). - Да. А теперь оставь меня.

3.2). - Идиот, ты хоть знаешь, кто я такой? Меня...

- Плевать, как там тебя папаша называет. От тебя требуется лишь одно - заруби себе на носу, что принимаю решения и отдаю приказы здесь я. И плевать, кто там о тебе хлопотал, будь ты хоть наследником Императора - здесь я указываю, что, как и кому делать.

3.2.1). по 1.1

3.2.2). по 1.2

3.2.3). по 1.3

3.3). - Меня не интересуют твои полномочия. Я буду делать то, что сочту нужным. А теперь пошел прочь.

- Олух... Мне не нужны открытые конфликты, но если ты полезешь не в свои дела... Дороги нынче неспокойны, часты несчастные случаи.

3.3.1). - Ты мне угрожаешь?

- Не будь глупцом, искренне советую. До встречи.

3.3.2). - Самодур брехливый... Терпеть не могу идиотов. Защищайся.

- Ты с ума сошел? От этого путешествия, быть может, зависит существование империи, я не могу допустить драки...открытого столкновения...между нашими людьми, а потому не приму твой вызов. Доброго здоровья, хех...

---

-Странный он, этот Бондри Слей...

---

\section{Незнакомство}

[через пару дней]

- Добрый день. Позволите отвлечь?

- Слушаю.

- Прежде всего, разрешите представиться. Муррен. Муррен Клач, к вашим услугам.

1). - А коли мне услуги пока не нужны? Да и слуги тоже, если на то пошло.

- А вот тут, прошу простить великодушно, ошибаетесь. Слуги-то верные, помощники, они везде нужны. Найти их, вот то да, сложнее. Впрочем, батюшка ваш, не поспоришь, как никто другой умеет. Да, и раз речь о нем... Я-то, сказать по правде, к вам по его просьбе, читай, приказу, приставлен, следить и охранять от случайностей, которых многовато происходит в последнее время.

1.1). - Что-то ты недоговариваешь...

- Недоговариваю? Отнюдь, от вас мне скрывать нечего. Я-то, по правде говоря, как раз таки пришел...расширить вашу осведомленность, так скажем.

- Внимательно слушаю.

- Мда-с... С чего бы начать историю, в которой, сколько уже распутываю, а начало никак не найду... Попробую быть лаконичным.

\chapter*{Заметки на полях}

\section*{Локации}

\begin{enumerate}
\item Императорский дворец
\item Гильдия магов
\item Гильдия наёмников
\item Гильдия торговцев
\item Гильдия убийц (негласная :)
\item Школа магов
\item Дом наёмников
\item Арена
\item Рынок
\item Главная площадь
\item Постоялый двор
\item 3-4 трактира
\item Кузница
\item Лавка целителей
\item Лавка кузнеца
\item Дом городского старосты
\end{enumerate}

\section*{Спискота --- синонимы}

\begin{itemize}
\item Сила, Мускулы, Атака
\item Ловкость, Сноровка, Проворство
\item Реакция, Скорость
\item Адреналин, Берсерк, Ярость
\item Выносливость, Неутомимость, Стойкость
\item Интеллект, Ум
\item Внимательность, Мудрость, Чуткость
\item Харизма, Обаяние, Привлекательность, Одаренность
\item Лидерство, Первенство
\item Сообразительность, Догадливость, Изобретательность
\end{itemize}

\section*{Черновик за авторством лично m1kc'a}

Первая встреча с Керри.

--- Я глава гильдии убийц!

1. О Боже! Сволочи, уроды, убийцы, мочить вас в сортире надо!

--- Не делай поспешных выводов. <экскурс>

1.1. Козлы, я вам не верю! Вы убиваете людей!

--- Да. Как жаль, что мы не можем убить и тебя...

1.2. Ладно, забей, что там насчет бати?

2. Вот как?

--- Я вижу, ты удивлен.

--- Да, еще бы, батя отправил меня прямиком к убийцам.

--- Заткнись и слушай...

3. Вот, значит, как...

--- Ты совсем не удивлен.

3.1. И почему вы все еще на свободе? 

--- <экскурс>

3.2. Не, ну а что, жрать-то надо. Так что там с батей?

\section*{Немного о навыках}

[при любом течении диалога]

- Прежде всего тебе необходимо усвоить некоторые навыки, без которых в нашем деле обходиться весьма непросто. Согласен, ты неплохо умеешь размахивать железом. Но искусство боя владением мечом отнюдь не ограничивается. Тем более, что и мечом ты сражаешься несколько...э-э...топорно и примитивно. 

1). --- Быть может...

--- Итак, чем хочешь заняться?

--- Хочу научиться лучше владеть мечом.

--- Хочу научиться бою в тяжелых доспехах.

--- Хочу научиться стрелять из лука

--- Хочу научиться владеть арбалетом

--- Хочу научиться использовать щит в бою

--- Хочу научиться быстрому бегу

2). --- Ну-ну... Проверим?

--- Ха-ха. Полегче, не горячись. Я отнюдь не хотел тебя оскорблять. Да и рановато тебе со мной тягаться.

--- Так-то лучше.

--- Итак, чем хочешь заняться?

--- Хочу научиться лучше владеть мечом.

--- Хочу научиться бою в тяжелых доспехах.

--- Хочу научиться стрелять из лука

--- Хочу научиться владеть арбалетом

--- Хочу научиться использовать щит в бою

--- Хочу научиться быстрому бегу

\end{document}
