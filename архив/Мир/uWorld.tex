\documentclass[12pt,a4paper]{book}
\usepackage[utf8]{inputenc}
\usepackage[russian]{babel}
\usepackage{amsmath}
\usepackage{amsfonts}
\usepackage{amssymb}

\title{\textbf{uWorld}}
\author{Автор: АндРЕЙ \and Координатор: m1kc}

\begin{document}

\maketitle

\chapter*{Пролог}

Эрия... Именно так называется мир, о котором пойдёт речь. Это огромный материк, на котором живут перворожденные, урук-хай и (куда же без них) люди. О перворожденных ни мне, ни расе людей почти ничего не известно. Народу урук-хай тем более, орки издревна мало интересовались окружающими, питая какую-то непонятную неприязнь к чужакам, как они называли людей и перворожденных. Надо сказать, что чужаками были скорее они, так как по хроникам эльфов орки лет на 500 позже людей высадились на юго-западном побережье Линары. И сразу, даже не успев обжиться на новом месте, объявили войну эльфам --- якобы глас богов повелевал уничтожить презренных остроухих. Как не странно, люди поддержали их, обратившись против своих наставников. Что было тому виной --- алчность ли правителей, охочих до сокровищ перворожденных, желание ли показать мощь новоявленной имперской армии? Доподлинно неизвестно, да и теперь уже неважно. Эльфы не смогли долго сдерживать объединённые силы людей и урук-хай и отступили в глубину Вечных лесов. Отряды людей (а урук-хай сразу сказали, что носу не сунут в эти проклятые Великими заросли), решившиеся преследовать перворожденных, бесследно исчезли, назад не вернулся ни один воин. Тогда было принято решение прекратить бесплодные атаки и выставить долгосрочные сторожевые заставы. Мол, пусть сидят и не высовываются. И на долгие века люди забыли о перворожденных... Да и с орками стали общаться гораздо меньше, торгуя лишь по большой необходимости. Тем более, что по окончании войны с эльфами империи пришлось заняться собственными внутренными проблемами --- высокородные лорды раздирали государство на куски, народ роптал, осознав бесполезность войны, унесшей тысячи людей, да и церковь, основательно прижатая прошлым правителем, вновь набирала силу, что было чревато большим бунтом.

Урук-хай тоже постарались обособиться от людей --- слишком разной были культура и религия. Да и общественное устройство орков, советы кланов, старейшины, руководствующиеся не столько политическими соображениями, сколько <<волею богов>>, непонятно (с точки зрения имперцев, разумеется) в чем проявляющейся, были слишком чужды людям.

И понеслись дни, месяцы, века... Империя стала монархией, потом развалилась на клочки уездов, потом опять монархия, тирания, демократия, монархия...

А вот здесь, пожалуй, стоит остановиться, ибо после случилось событие, изменившее дальнейшую судьбу огромной страны. Надо сказать, что люди до оного времени не имели ни малейшего понятия о магии, и магов среди них не было. Да, в давние времена видели, как перворожденные исцеляли людей, но сами ничего подобного не умели. И вдруг через 800 эльфийских циклов после войны (7000 лет) объявился человек, заявивший, что владеет магическими способностями. Поначалу, разумеется, его подняли на смех и даже чуть не изваляли в дёгте. Тогда он превратил стоящий на площади дом в цветущее дерево, подобного которому никто и не видел. Впрочем, он недолго показывал фокусы на потеху толпе. Уже на следующий день император Та-Реми, выйдя на трясущихся ногах на балкон в сопровождении мага, объявил во всеуслышание, что отказывается от всех прав на престол в пользу этого самого мага, Ремана I. А тот сообщил ошеломленным жителям Тантикула, столицы империи, что вводит новый тип государственного устройства --- ненаследственную монархию, суть которой заключается в том, что преемника себе правитель находит, ориентируясь лишь на его моральные качества. Правителем может стать кто угодно --- сельский староста, старый библиотекарь, заплывший жирком купец, хилый воин или бравый вояка, служка из соседнего кабака. И подданные не будут знать имени императора, из какой он семьи и кем он был до коронации --- словом, ничего о его прошлой жизни. Личность правителя становится известна лишь после его смерти. Примерно такой была суть пространной речи новоявленного императора.

Собравшиеся на площади люди поначалу ничего не поняли, но единодушно решили, что по такому поводу можно и напиться.

Впрочем, простой народ с облегчением вздохнул уже через пару месяцев --- стали открываться новые лечебницы и приюты, снизились налоги. Хотя главным событием было, бесспорно, другое --- новый император объявил, что некоторые люди способны творить магию, и открыл школы, которые занимались набором и обучением наделённых даром детей. Через какие-то двести лет никто уже не мог представить обыденную жизнь без магии, относясь к ней, как к чему-то совершенно естественному. Другое дело церковь. Первосвященник объявил магов пособниками дьявола и запретил всякое общение с ними. Но гнев Благих --- это ещё когда будет, а польза от магии --- вот она, под носом. Церковники быстро поняли это и на время затихли. Жизнь в империи потекла своим, хотя и новым, чередом.

\chapter{Линара (эльф. <<освящённый>>)}

Самый большой континент. Основное население --- эльфы, орки, люди.

\section{География}

\subsection{Тревожный залив}

Когда-то корабли, идущие в северную часть Катора, обходили его с восточной стороны. Но несколько столетий назад суда перестали возвращаться. Наверное, мы бы никогда и не узнали, что с ними случилось, если бы не капитан Тракк. Однажды этого старого морского волка застали в трактире в совершенно нетипичном для него состоянии. Бравый моряк стоял у барной стойки, ссутулившись, и допивал четвёртую кружку эля; вся его одежда была изодрана, а лицо обезображено основательным багровым шрамом. С трудом набив дрожащей рукой трубку, капитан очень тихо, иногда делая долгие паузы или срываясь в плач, рассказывал посетителям трактира об ужасном морском чудовище, поднявшемся из глубин и проглотившем весь его корабль вместе с командой. Сперва ему никто не поверил, но проходили десятилетия, а немногие смельчаки, отправляющиеся на Катор этим путём, по-прежнему не возвращались. И постепенно за заливом закрепилось название Тревожный...

\subsection{Шиоканский мыс (эльф. <<шиока>> - склоняющий к размышлению)}

Надо сказать, что у эльфов особое, трепетное отношение к океану, некоторые считают его огромным, живущим по своим законам существом. Большинство же просто любит созерцать бесконечную гладь воды, \mbox{думать} о бесконечной спирали жизни, плавно уводящей материальные сущности на новые витки бытия, не останавливаясь ни на секунду, делая смыслом жизни движение и придавая движению смысл, воссоединяя бытие и небытие и разделяя их вновь, создавая энергию абсолюта, в котором нет ничего, кроме кроме постоянного колебания двух стихий, в котором любая их частица становится иллюзорной, обретая существование лишь в постоянном круговороте этого непрекращающегося движения — ну и тому подобной херне. И живший когда-то здесь отшельник, имя которого история не донесла, так назвал это место в своих мемуарах. Имя эльфа забылось, а его случайная мысль осталась в веках.

\subsection{Мыс Киоа Лара (эльф. <<вольная птица>>)}

История происходжения сего названия очень красиво и поэтично описана в 397-страничной <<Береговой метрике>>. Если вкратце и прозаично, то этот мыс донельзя засижен всевозможными морскими птицами. Здесь невозможно спокойно посидеть и час без того, чтобы ваш наряд не испачкали. И известный поэт, побывавший здесь, после непродолжительных творческих мук нарёк сие прекрасное место <<Киора лара>>, мыс вольной птицы.

\subsection{Пролив Яогран}

Одни говорят, что Яогран --- имя великого мага, участвовавшего в создании Лаша, другие --- что это искажённое название магических чудовищ, Аогров, призванных магами то ли шутки ради, то ли для охраны водного пространства вокруг Лаша... Возможно, верно и то, и то; возможно, ответ кроется совсем в другом. Но одно известно наверняка --- название как-то связано с Лашем.

\subsection{Залив Надежды}

Первое географическое название, данное людьми на Линаре. Около семи тысячелетий назад изрядно потрёпанный и почти исчерпавший запасы провизии флот людей подошел к Линаре. Поначалу, когда смотровые только-только заметили очертания каменных утёсов на скалистом полуострове, восторгам не было предела. Но через пару миль стало ясно, что пристать к берегу в этом месте невозможно --- штормящее море и обилие рифов обрекали смельчаков (и идиотов) на гибель. Решено было обходить остров с восточной стороны. И очень скоро адмиралы натолкнулись на тихую, спокойную бухту, названную после удачной высадки заливом Надежды.

\subsection{Скалистый полуостров}

Это место --- первое, что увидели люди на необьятной Линаре. Название говорит само за себя: весь полуостров --- это огромные скалы, кое-где перемежающиеся каменистыми равнинами. Здесь было решено основать первое на Линаре поселение людей, позднее ставшее столицей и крупнейшим городом.

\subsection{Море Наваждений}

Здесь не бывает крупных судов, выходящих в открытое море, но небольшие рыбацкие тендеры и шлюпы в непогоду порой уносит на десятки и даже сотни миль от берега, и счастливчики, сумевшие перебороть стихию, после рассказывают о каких-то смутных тенях за бортом, странных звуках, похожих на голоса, хотя на десятки кабельтов вокруг лишь водная гладь...

\subsection{Зеркальный пролив}

Узкий пролив между Линарой и Дарсу. Холмистый Кражский полуостров находится значительно выше уровня моря, то же самое можно сказать и о земле западнее Кронта. Поэтому между ними, в проливе, практически не бывает ветров, течение здесь спокойное, и гладь воды лишь иногда слабо колышется от проплывающей у поверхности рыбёшки. И если вы попадете на побережье в один из тех немногих часов, когда в проливе нет судов (а это бывает нечасто), то сможете полюбоваться как на облака под ногами, так и на свою физиономию. Вероятно, именно из-за спокойной воды пролив и был назван Зеркальным.

\subsection{Вааржский полуостров}

Заболоченный полуостров осваивали на протяжении полутора веков, но до сих пор есть места, куда ещё ни разу не ступала нога орка. Причина отчасти в ужасном климате, отчасти в зловонных болотных газах. Многих первопроходцев останавливали дикие звери, среди которых были и одни из самых свирепых хищников --- ваарги, известные своей силой и выносливостью. Первые поселенцы не без причин опасались этих животных, про них рассказывали десятки историй, тысячи небылиц, словом, это была главная тема для разговоров среди местных жителей. Постепенно вааргов практически полностью истребили, но в застольных разговорах они превратились из опасных зверей в исчадия ада; сам же полуостров стали сначала в шутку, а потом просто по привычке называть Вааржским.

\subsection{Причал Северного ветра}

Довольно большая пристань, связующее звено между Катором и Линарой. Возле нее образовалось целое небольшое поселение торговцев, обменивающих товары, моряков, отдыхающих после рейса... Кстати, именно простые матросы и придумали название этой пристани --- Причал северного ветра. Дело в том, что в сторону Вааржского полуострова почти всегда дуют с севера сильные ветра. Проклинаемые моряками, идущими на Катор, эти ветры горячо благославляются ими же по дороге обратно --- работы хватает только рулевому. В честь силы, способствующей их возвращению за трактирную стойку, и назвали причал эти бравые ребята.

\subsection{Кретчский пролив}

Через пролив пролегает главный (и, в общем-то, единственный) торговый путь, соединяющий Линару с Катором. Но помимо торговых судов, много здесь и рыбацких посудин --- только в этом проливе в изобилии водится вкусная рыба, считающаяся изысканным и аристократическим (ибо продают её втридорога) лакомством. Знающие люди поговаривают, что она пресноводная, и по-видимому, выплывает из какой-то подводной пещеры, что в общем-то, подтверждают и рыбаки, способные заболтать любого бедолагу россказнями о единственном месте и верном часе, в который только и можно поймать эту рыбину. По её названию, Кретч, и дали имя проливу.

\section{Города орков}

\subsection{Прэт}

Город-крепость на Вааржском полуострове. Довольно большой и оживлённый по орочьим меркам, важное... да что уж там, единственное связующее звено между Катором и Линарой. Осёдлое население --- различного рода торговцы и жулики, которых весьма сложно отличить друг от друга, лесорубы и охотники, промышляющие в Вааржском лесу, разумеется, гарнизон крепости, весьма многочисленный --- хоть остроухие и отделены проливом, но мало ли... В Прэте есть небольшая школа, созданная здешним верховным шаманом --- Зигром. Правда, люд ей не слишком \mbox{доволен ---} ремесленников и мастеровых в неё не допустили, а стать шаманом --- мечта далеко не каждого орка, ибо почёт велик, но и ответственность огромна.

\subsection{Трендор}

Когда-то здесь было крохотное поселение рыболовов, которое стремительно стало расти по мере освоения Катора и развития Прэта --- через него проходил кратчайший (и единственный более-менее безопасный) тракт из столицы на Вааржский полуостров, а учитывая расстояние и обилие диких животных, необходимо было спокойное защищённое место с хорошим пивом и мягкими кроватями. Сейчас это небольшой городок с обилием разного рода трактиров, лавок и, вероятно, лучшей на Линаре воинской школой. Оно и немудрено --- в здешних местах молодым воинам и охотникам хватает как практически безопасных живых мишеней, так и зверей, которые сами учатся охоте на начинающих вояках...

\subsection{Кронт}

Столица орков. Самый большой и самый разнородный город урук-хай. Здесь в уличной давке можно встретить кого угодно --- верховного шамана, молодого циркача, ловкого воришку (пожалуй, самая разорительная из возможных встреч). Хотя, попадись вы в цепкие лапы банкира, могущего увлечь любого <<дармовым>> золотом взаймы --- пожалуй, потом позавидуете обокраденному... В Кронте находится множество школ, ремесленных мастерских, с охотой берущих учеников --- так что если вы готовы добывать знания п\'{о}том и кровью --- вам сюда. Если вы орк, конечно.

\subsection{Мроин}

Небольшой городишко --- не то застава для защиты от возможного нападения людей, не то торговый город, способствующий установлению контактов с ними. Живёт в Мроине самый странный и разнопёрый люд --- прежде всего, ясное дело, ловкие торгаши, получающие огромные барыши от торговли с людьми, но, кроме того, здесь много хороших воинов, умудрённых опытом шаманов, кузнецов и пройдох из тайного сыска. Последних, конечно, везде хватает, но очень уж многих неосторожных торговцев, нарушающих пакт об ограничении торговли с людьми, находят в оврагах и на излучинах Нарвы.

\subsection{Пактл}

Город, лишь немногим уступающий Кронту по количеству жителей, нечистот и трактиров. На северо-востоке густые, непролазные, но богатые природными дарами (а также разбойным людом) леса. По мнению любого орка (кроме жителя Пактла, разумеется), это пристанище отчаянных сорвиголов, жулья и ещё черти знают кого. А один из уроженцев Пактла охарактеризовал свой город как <<...единственное место, где любой орк может добиться почета и уважения без необходимости угождать шаманам>>. Надо сказать, что большинство сыскарей верховного шамана в этом городе погибает от несчастных случаев в течение одной-двух недель после появления, и это одна из главных причин популярности этого места.

\section{Города людей}

\subsection{Плес}

Аналог Мроина с небольшой поправкой на человеческий менталитет. Хоть огромные и не слишком обаятельные орки совсем рядом, жители отчётливо понимают, что вряд ли они ни с того ни с сего нападут --- текущее положение вещей всех устраивает. Поэтому и вояки не слишком трясутся за безопасность города, и торгаши имеют смелость порой кидать орочьих купцов на большие деньги, и ребятишки толпами дразнят огромных и вроде бы свирепых орков... Но город всё же пограничный, и много, ох как много здесь может быть неприятных сюрпризов для беспечного и чересчур болтливого путника.

\subsection{Приад}

Посёлок на месте древнего эльфийского городища. Заурядное провинциальное местечко с серыми невзрачными домишками, изобилием свиней и заплёванной главной (а по большому счёту --- единственной) улицей. Яркими пятнами на этом фоне выглядят редкие, но роскошные особняки магов, приближённых к монарху --- по слухам, это уникальное для чаровников место, пропитанное силой и ещё чёрт знает чем. Жителей все это не особо интересует --- есть за чей счёт питаться --- отлично. Есть на кого работать --- тоже неплохо. На востоке находится довольно глубокая шахта, разрабатываемая в рыхлой и совершенно пустопородной земле --- ещё одна из многих странных прихотей местных господ.

\subsection{Шиут}

Очень крупный охотничий лагерь. Образовался сравнительно недавно, около ста лет назад, когда изобилие зверей между Приадом и Тантикулом стало мешать нормальному сообщению между городами. И если большинство животных было сравнительно безопасными, то сирумы --- похожие на медведей, но несравнимо более ловкие и хитрые звери, являлись настоящей бедой не только для странников-одиночек, но и для дорожных патрулей. Тогда указом монарха и был создан этот... кочующий палаточный посёлок, который то разрастается, то уменьшается, но всегда остаётся примерно на одном и том же месте.

\subsection{Тантикул}

Столица людей. Город, олицетворяющий собой упорство и трудолюбие его строителей --- более ста лет строилась это громада, огромная крепость, вырубленная в скале. Большая часть крепостных стен --- цельный гранит, остаток горного хребта самой высокой горы Линары --- Спасительной (именно её увидели смотровые с мачт ищущих сушу кораблей). Если верить здешним купцам, то это --- самое весёлое место на Земле с самыми качественными товарами, самым выдержанным вином, самыми... эээ... лучшими дамами и прочее, прочее, прочее. На деле же Тантикул --- неспокойный муравейник с беспрестанными солдатскими заварушками, толстыми домохозяйками и выливающимися вам на голову помоями. Хотя, никто не спорит, здесь есть и отличные мастера, и уникальные товары, и обученная регулярная армия. Если вы ищете знания --- милости просим, различные гильдии, коих в этом месте великое множество, к вашим услугам. Хотя попасть в них не всегда бывает просто. В общем... заходите.

\subsection{Диовен}

Впечатляющая своей мощью и монолитностью застава. Здесь состредочено большое количество солдат империи --- первый город на пути к великому лесу. Правда, дисциплина уже не та, что семь тысячелетий назад --- Уши, специальная служба монарха, аналог тайного сыска народа урук-хай, всё чаще в последнее время сообщает, что вояки давно уже морально разложились, обрюзгли, да и вообще вряд ли в случае надобности натянут кирасы на раздобревшие животы. Но, разумеется, здесь остались и настоящие вояки, и отличные кузнецы, и несколько хороших наставников...

\subsection{Тропиус}

Небольшое поселение рыбаков, охотников, лесорубов и скотоводов. Здесь очень благоприятное для жизни место --- горячие ключи даже в самые суровые зимы поддерживают в заливе плюсовую температуру, а значит, для рыбаков и ловчих здесь настоящее раздолье. В отличие от Диовена, в Тропиусе в общем-то нет регулярного войска, но жители и сами не промах --- жизнь в изобилующем церапторами и медведями месте совсем не располагает к неловкости и лени.

\section{Города эльфов}

\subsection{Тианталь}

Форпост эльфов, когда-то небольшое укреплённое поселение, сегодня --- разросшийся городок, окруженный цепью башен и укреплений. Река Сориав, очень широкая здесь, которая разбивает город на две части. Некоторые дома, построенные на сваях, <<торчат>> из воды вдали от берега и мостов --- жители добираются сюда на лодках. В Тиантале много рыбаков и ремесленников, торговцев. Неподалеку от города находятся каменоломня и небольшая верфь, правда, каменоломня практически выработана.

\subsection{Приора}

Единственный оставшийся <<традиционным>> эльфийский город --- растущий не вширь, а вверх --- поселение на могучих, древних деревьях, калибах. Здесь живут самые консервативные эльфы, не желающие иметь ничего общего с презренными людьми и орками. Вам придётся очень долго перебираться с дерева на дерево, пытаясь найти кузнеца или какого-нибудь гончара. Население города --- практически исключительно охотники, маги, учёные --- цвет эльфийской культуры. Стоит оговорится, правда, что кузнеца вы при желании найдёте --- одного из тех немногих, кто признает лишь магическую обработку стали --- но стоят его произведения (а иначе их и не назовёшь) очень и очень много.

\subsection{Хорам}

Очень старый город, сердце эльфийской науки и магических изысканий. В Хораме несколько знаменитых институтов, обширнейшие библиотеки, самая большая кузнечная община. Находящийся в стороне от торных путей, практически не производящий товары, а значит, тихий и немноголюдный, этот город --- лучшее место для тех, кто решил посвятить свою жизнь науке или магии. Ну или для тех немногих, кто хочет просто пожить спокойно и без суеты, вдали от соблазнов и ловушек большого города, скупо растрачивая свои небольшие сбережения.

\subsection{Жуденир}

Крупный торговый город, через который проходит большая часть дорог с востока на запад. Пройти сотню шагов и не наткнуться на трактир здесь можно лишь в том случае, если вы только что из него вышли и теперь, мертвецки пьяный, ходите по кругу. Собственно, весь город --- причудливая комбинация трактиров, постоялых дворов и обширных рынков, на которых можно найти всё, что душе угодно. Наверное, единственное, что невозможно купить здесь ни за какие деньги --- тишина и уединение. Даже запрись вы в самой дорогой комнате для постояльцев, вам всё равно не укрыться от монотонного шума большого города.

\subsection{Леораль}

Небольшое поселение охотников, рыболовов и ремесленников. Кроме того, здесь находится очень хорошее воинское училище и единственная на эльфийской части континента боевая арена, которую нередко посещают выпускники, а чаще --- ученики (хотя им это строжайше запрещено) этого самого училища. Впрочем, и иной люд сюда захаживает, но нечасто. Как ни странно, хотя поселение и находится на самом крупном торговом пути между столицей и западными городами, но здесь не слишком много торговцев и прочих жуликов. Вероятно, потому, что обилие диких зверей, заболоченная местность и, как следствие, частые тяжёлые болезни делают здешнюю жизнь не слишком привлекательной.

\subsection{Рибин}

Столица эльфов, город на лесистом побережье великого океана. Хотя внешне он и не похож на крепость, но густые заросли вокруг городских стен очень напоминают монолитное заграждение --- не зная дороги, пройти по зарослям, кишашим странными, большей частью магически созданными существами, очень и очень непросто. Это, как и многое другое в городе, заслуга здешних магов, вероятно, самых мудрых и сильных (хотя кто знает, как измерить силу мага...) на Линаре. Как и любой крупный город, Рибин весьма разнороден --- рабочие, воины, алхимики, наёмные убийцы --- кого только не встретишь на здешней площади...

\chapter{Дарсу}

Появился из океана через некоторое время после прекращения контактов между людьми и орками неподалеку от орков. Несмотря на близость, континент мало исследован народом урук-хай, отчасти из-за сильной его заболоченности, отчасти из-за влажного климата, вызывающего странные болезни даже у самых закаленных воинов, но в большей мере из-за огромного количества диких животных и... Многое поговаривают о обитателях Дарсу. Странное место. Людям и эльфам, вероятно, почти ничего о нем не известно. Население очень невелико, в основном это сосланные за провинности орки, всякий странный народец - лекари, алхимики, которые бог знает что там ищут. Ну и люди, невесть как сюда забредшие - то ли скрываются от кого-то, то ли ищут что-то. Впрочем, по образу жизни они в большинстве своем много ближе оркам, чем людям.

\section{География}

\subsection{Лафров залив}

\underline{По чьей-то вине здесь нет описания.}

\subsection{Кражский полуостров}

\underline{По чьей-то вине здесь нет описания.}

\chapter{Катор}

Большой остров неподалеку от земли урук-хай. В отличие от Дарсу отлично изучен орками и густо заселен за исключением северной его части, гор с вечными снегами и суровым холодным климатом. Там живут странные звери - не звери, то ли каменные, то ли ледяные существа, неловкие, медленные, но невероятно сильные. Надо сказать, что хотя орки с Линары тесно общаются со своими сородичами здесь, они очень разные. Здешние занимаются в основной добычей и обработкой даров недр земли, мало охотятся, да и вообще презирают другие ремесла, считая их чуждыми для истинных орков, хотя есть и исключения. Нельзя не заметить, что долгие часы, проводимые под землей наложили на многих свой отпечаток - каторцы в большинстве своем невысокие (по меркам урук-хай, разумеется), сутуловатые, плохо видящие при ярком свете, но очень искусные в своем деле орки. Из их рук вышло самое лучшее из виданного когда-либо людьми оружие.

\chapter{Миритал}

Хотя этого никто уже, вероятно, и не знает, но именно отсюда тысячелетия назад армады людей двинулись в поисках нового дома на север, в сторону Линары. В те времена континент медленно (по меркам суетливых людей) уходил под воду, и несмотря на все усилия миллионов рабочих, беспрестанно строящих и обновляющих колоссальные дамбы и заграждения, катастрофа была неизбежна. Люди метались в невиданной панике, спешно строили корабли, лодчонки, плоты... Спасти удалось лишь немногим - правителям и их приближенным, многим матросам, некоторым счастливчикам - ловким и сильным людям, всеми правдами и неправдами стремившимися к единственной цели - выжить, а значит, попасть на корабль.

Оставшиеся миллиарды людей погибли. Не сразу, долго длилась агония цивилизация. Еще пару лет жизнь теплилась в горах и на возвышенностях, но за эту пару лет оставшиеся потеряли людской облик и совершенно озверели в каждодневной борьбе за жизнь.

Миритал исчез под водой. Но спустя примерно пять тысяч лет вновь поднялся над уровнем океана. Никому не известно почему, да это и не самое важное. Пахнущая гниющими водорослями, покрытая причудливыми растениями земля, возрожденный Миритал тоже заселен. Но не птицами или животными, нет. Души самых озлобленных, горящих ненавистью людей оказались навечно привязанными к этому месту, и непонятная, беспредметная злоба не дает им уйти.

\section{География}

\subsection{Троирский полуостров}

\underline{По чьей-то вине здесь нет описания.}

\subsection{Залив Корба}

\underline{По чьей-то вине здесь нет описания.}

\subsection{Риольский полуостров}

\underline{По чьей-то вине здесь нет описания.}

\subsection{Залив Спокойствия}

\underline{По чьей-то вине здесь нет описания.}

\subsection{Болотный полуостров}

\underline{По чьей-то вине здесь нет описания.}

\chapter{Лаш (эльф. <<вознесённый>>)}

Созданный после поражения в войне великими магами эльфов <<остров>>, висящий над океаном неподалеку от северной части Линары. Там живут кланы эльфов, наотрез отказавшихся оставаться в сравнительной близости к ненавистным людям и урук-хай. Надо сказать, что ко времени этой истории, между отдельными кланами начались сначала свары, а потом и настоящие небольшие войны, из-за чего эльфы Линары стали как своих сородичей, так и этого места.

\chapter*{Заметки на полях}

\begin{itemize}
\item Море Наваждений: Моряки часто видят какие-то тени.
\item Ваараг --- орочий бог ветра.
\item Краг --- большой паук.
\item Лафров залив: Лафр --- орк. Здесь его дом и маленькая бухта.
\item Корб --- адмирал, живший 7000 лет назад.
\item Залив Летнего Дыхания: Здесь всегда тёплая вода.
\end{itemize}

\end{document}